\chapter{Fundamentos conceituais}
  \section{Manifesto Ágil}
    No começo do desenvolvimento de \textit{software}, a metodologia mais utilizada
    para realizar a produção de \textit{software} era a \textit{waterfall}, que
    consiste em um desenvolvimento faseado: primeiro especificamos todo o sistema,
    depois desenvolvemos, testamos e por fim colocamos em produção. Muitas vezes
    quando terminávamos de especificar o sistema todo, as necessidades do
    negócio haviam mudado, gerando uma mudança no desenho do sistema e fazendo a
    equipe voltar para a fase de especificação. O que também acontecia, era que
    durante o desenvolvimento, as necessidades do negócio mudavam novamente e
    a equipe precisava fazer a escolha de voltar para a fase de desenho ou continuar
    o da forma que foi desenhado. Esta decisão não era fácil e gerava muitos conflitos
    entre a área de negócio e a equipe de desenvolvimento, pois muito do que foi
    realizado teria que ser refeito. \newline
    Para solucionar este problema em fevereiro de 2001 foi publicado o manifesto
    ágil, um conjunto de valores e princípios para auxiliar que o desenvolvimento de
    \textit{software} seja realizado de forma rápida e adaptativa às necessidades
    do negócio, onde é mais enfatizada a necessidade de comunicação entre as pessoas
    do que o planejamento excessivo, a adaptação a mudanças ao invés de soluções
    definitivas e a priorização da realização do produto ao invés de seguir processos
    e ferramentas \cite{manifestoAgile}. \newline
    Conforme apresentado no Agile Principles, Patterns, and Practices in C\#
    \cite{martin2007agile}, o manifesto ágil valoriza as pessoas que estão
    no desenvolvendo do \textit{software}, mais do que processos e contratos.
    Não é necessário ter desenvolvedores extremamente experientes, mas sim
    uma equipe que esteja unida e que seja multidisciplinar, ou seja, que além de
    desenvolvedores, é necessário manter as pessoas com entendimento sobre o
    negócio para que durante o desenvolvimento, dúvidas e mudanças no negócio
    possam ser explicadas e a equipe consiga desenvolver um produto de maior
    valor para o negócio. É importante notar que, sob a visão ágil, mudanças
    nos requisitos do projeto são bem-vindas, pois o produto que está sendo
    construído é mais importante do que documentos e processos, é mais
    importante atender as necessidades do negócio do que seguir um planejamento,
    ou seja, embora seja importante planejar, precisamos estar preparados para
    mudanças e imprevistos. Desta forma, os desenvolvedores devem construir
    um sistema que seja fácil de alterar, possa ser alterado conforme as necessidades
    do negócio. \newline
    Com esses valores e princípios, a produção de \textit{software} se tornou
    mais rápida, o que antes demorava anos para ser produzido começou a ser
    realizado em meses, pois o sistema começou a ser desenhado conforme era
    desenvolvido, os desenvolvedores não desenhavam primeiro todo o sistema
    para só depois começar o desenvolvimento, ao invés, eles já começavam a
    desenvolver e documentavam somente o necessário para facilitar a comunicação
    entre a equipe, desta forma, a cada mudança que ocorre, não é mais necessário
    desenhar novamente toda a solução, mas sim adaptar o que já foi desenvolvido. \newline
    As metodologias advindas das práticas ágeis, promovem o conceito de iterações,
    que consiste, em um curto espaço de tempo, normalmente semanas, seja produzido
    \textit{software} em produção, para que desta forma consigamos avaliar os impactos
    do que foi contruído para o negócio em menos de um mês e seja possível já mensurar
    o valor que está sendo gerado. Como o tempo de desenvolvimento é curto, está
    previsto que vamos errar diversas vezes, as soluções que vamos produzir não
    vão atender todas as necessidades do negócio, porém é através do aprendizado
    destes erros que vamos entendendo as necessidades do negócio e corrigindo o
    sistema para que ele se torne cada vez mais aderente a essas necessidades,
    temos que errar o mais rápido possível para que possamos aprender com os nossos
    erros e acertar o mais rápido possível.

    \section{\textit{DevOps}}
      Através dos valores e das práticas ágeis e com o surgimento de novas
      tecnologias, mais práticas foram sendo criadas e novas ferramentas surgiram
      para auxiliar o desenvolvimento de \textit{software}, já com a perspectiva
      ágil. Desta forma, surgiu o conjunto de práticas denominado \textit{DevOps},
      que consiste em auxiliar a equipe de desenvolvimento e de operação a trabalharem
      de forma unida \cite{TheDevOpsHandbook}. Conforme o tempo de desenvolvimento
      dos produtos foi encurtando, surgiu a necessidade de realizar o \textit{deploy}
      em produção, diversas vezes ao longo do dia, pois como trabalhamos de forma
      adaptativa e valorizamos o produto sendo estregue de forma rápida, para gerar
      valor o mais rápido possível, a infraestrutura montada e a equipe de operações,
      precisam estar preparados e embora seja importante realizar entregas rápidas,
      ainda é necessário avaliar a qualidade do que está sendo entregue, para
      evitar erros em produção e não tornar o que deveria gerar valor, um problema
      para o negócio. \newline
      As práticas de \textit{DevOps}, nos fornecem um conjunto de ações que podemos
      realizar para que consigamos evitar que os erros aconteçam e que, caso venham
      a acontecer, teremos as informações necessárias para avaliar o ocorrido e
      resolver da melhor forma o mais rápido possível, neste capítulo vamos abordar
      rapidamente as principais práticas e durante a proposta elas serão exploradas
      mais detalhadamente. \newline
      Entre as principais práticas está o de \textit{code review}, que consiste
      que antes de um \textit{deploy} ocorrer, o código que foi produzido será
      verificado em busca de falhas, se está seguindo as boas práticas, erros de
      lógica, brechas de segurança, ou seja, qualquer coisa que possa causar algum
      erro em produção. Outra prática importante é a de estratégias de ambientes,
      quando estamos produzindo um \textit{software} não é aconselhável realizar
      alterações direto em produção sem antes ela ter sido testada e o código
      ter sido avaliado. Desta forma, é importante ter um ambiente separado de
      produção, onde seja possível os desenvolvedores realizarem testes das
      funcionalidades que estão sendo produzidas, para assegurar a qualidade do
      que foi desenvolvido e conseguir avaliar se está aderente a necessidade do
      negócio. \newline
      O \textit{DevOps} apresenta que o processo de desenvolvimento deve levar em
      consideração todos os processos de verificação de qualidade e efetivamente
      o \textit{deploy} em produção. A equipe de operação, que é responsável por
      realizar o \textit{deploy}, faz parte da equipe de desenvolvimento e o tempo
      que será utilizado para realizar o \textit{deploy} considerando o
      \textit{code review}, alterações na infraestrutura do sistema, o tempo do
      \textit{deploy}, análises de segurança, entre outro qualquer processo que
      deva ser realizado para assegurar um \textit{deploy} com qualidade, verificando
      a presença de possíveis \textit{bugs}, deve ser considerado durante o
      planejamento da iteração, e caso venha a aparecer algum ponto que deva ser
      ajustado pelos desenvolvedores, eles devem ser notificados o mais rápido
      possível, para que seja realizado o ajuste e o \textit{deploy} possa acontecer. \newline
      Para otimizar esta comunicação, e facilitar adicionar estes processos de
      verificação do código possibilitando o \textit{deploy} e realizar a manutenção
      de forma mais fácil e rápida, foram criados os conceitos de CI (\textit{
      Continuous Integration}), CD (\textit{Continuous Delivery}), TDD (\textit{
      Test-Driven Development}) e sustentabilidade. Existem outros concenitos dentro
      das práticas de \textit{DevOps}, vamos explicar somente os conceitos para
      auxiliar no entendimento da proposta do trabalho.

      \subsection{\textit{Continuous Integration}}
        Quando estamos trabalhando em equipes, é natural que em determinados momentos
        seja necessário que dois desenvolveres tenham que alterar o mesmo arquivo,
        se não trabalharmos utilizando o conceito de CI, só vamos descobrir esse
        conflito quando um dos dois desenvolvedores perceberem que o seu código
        foi sobrescrito. Uma das principais ferramentas utilizadas para auxiliar
        no CI são os sistemas de controle de versões (Git, Mercurial, SVN, etc.)
        que conseguem criar versões do seu código em sua máquina local e essas versões
        podem ser compartilhadas para um servidor, chamado de repositório (GitHub,
        Bitbucket, GitLab, etc.), e compartilhado com outras máquinas, apresentando
        os conflitos nos arquivos e possibilitando que o desenvolvedor realize as
        mudanças necessárias \cite{ProGit}. Outra grande vantagêm de utilizar o CI,
        é que sempre que alguém criar, editar ou exluír um arquivo, todos so membros
        da equipe conseguem replica esta alteração para as suas máquinas de uma forma
        automatizada. \newline
        Com a utilização de um sistema de controle de versões, devemos separar o
        nosso trabalho em \textit{branchs} \cite{TheDevOpsHandbook}, que possibilita
        separar as alterações que realizamos em outra linha de mudanças, onde
        o que for realizado nesta \textit{branch} não terá impacto na linha
        principal do código até que seja solicitado realizar o \textit{merge}
        (juntar as duas \textit{branchs}) \cite{ProGit}, desta forma garantimos maior
        controle sobre o que está sendo implementado em nossas funcionalidades, pois
        cada \textit{branch} deve tratar somente de uma implementação, podendo ser
        uma funcionalidade nova ou uma melhoria em uma funcionalidade já existente.
        Com as \textit{branchs}, somente disponibilizamos que o nosso código seja
        integrado com os desenvolvedores que estão atuando em outras \textit{branchs},
        depois de finalizado os nossos desenvolvimentos e realizar o \textit{merge}
        para a \textit{branch} principal. Desta forma evitamos integrar um código
        que pode impactar o desenvolvimento de outras pessoas na nossa equipe por
        causa de estar inacabado. \newline

      \subsection{\textit{Continuous Delivery}}
        Agora que conseguimos integrar o nosso código de forma que os conflitos são
        apresentados para nós e as implementações são realizadas em \textit{branchs}
        separadas, conseguimos aplicar essas implementações de forma contínua
        \cite{TheDevOpsHandbook}. Com o conceito de CD, podemos realizar a criação
        de um \textit{pipeline}. Para cada implementação, conseguimos realizar
        diversos testes, verificar cobertura de código e fazer o \textit{deploy}
        de forma automatizada, diminuindo as ações que a equipe teria que executar
        para cada demanda, consequentemente otimizando o tempo de entrega
        \cite{ContinuousDelivery}. Quando aplicamos CD para o nosso sistema,
        para cada automatização, precisamos implementar uma ferramenta diferente.
        Por exemplo, uma ferramenta de gerenciamento de dependências (por exemplo,
        Nexus) para verificarmos as dependências entre as implementações, uma ferramenta
        para a elaboração de testes automatizados (por exemplo, Selenium), para
        cada necessidade que o seu produto necessite é interessante pesquisar alguma
        ferramenta para realizar esta automatização e aplicá-la ao seu \textit{
        pipeline}. \newline
        Com o auxílio do \textit{pipeline} podemos aplicar estratégias de criação
        de ambientes separados para o processo de desenvolvimento. Com o intuito de
        possibilitar que os desenvolvedores consigam testar as funcionalidades que
        já eles estão desenvolvendo e os \textit{testers} testarem as funcionalidades
        que já foram desenvolvidas sem que as alterações dos desenvolvedores
        atrapalhem, é interessante criar ambientes separados para cada etapa do
        fluxo de entrega. \newline
        Cada projeto possui uma combinação diferente de ambientes, cada uma sendo
        a estratégia tomada pela equipe do projeto, onde o mais importante nessa
        criação é a de realizar entregas rápidas garantido a qualidade daquilo
        que foi desenvolvido e, caso venha a acontecer algum impacto negativo na
        implementação, podemos utilizar os ambientes para identificar e mitigar
        os erros antes que entre em produção \cite{TheDevOpsHandbook}. \newline

      \subsection{\textit{Test-Driven Development}}
        Para assegurar a qualidade do nosso \textit{software}, é necessário realizar
        diversos testes antes que uma funcionalidade seja disponibilizada em produção,
        eles devem ser considerados durante o período de desenvolvimento, todas
        as nossas funcionalidades devem ser testadas e os testes não devem somente
        cobrir o código criado, mas deve garantir que os requisitos da funcionalidade
        estejam sendo cumpridos. Desta forma, verificamos se o código está funcionando
        de forma automatizada \cite{CleanCode}. \newline
        Pensando nisso, foi criada a prática do TDD (\textit{test-driven development})
        \cite{TestDrivenDevelopment}, que consite em primeiro criar os testes, ver
        eles falharam, pois a funcionalidade ainda não foi desenvolvida, e conforme
        o desenvolvimento, ir verificando se os testes estão passando e se é necessário
        mais teste. Conforme vamos criando cenários de teste e verificando o resultado
        do nosso desenvolvimento, nos deparamos com diversos requisitos, e para
        garantir que eles estão sendo cumpridos, devemos adaptar os nossos testes para
        que estes requisitos sejam verificados. Caso haja uma mudança nos requisitos,
        podemos alterar algum teste que desenvolvemos e verificar se caso o teste
        continue passando, a funcionalidade desenvolvida está contemplando a mudança
        e caso o teste falhe, teremos que ajustar o nosso desenvolvimento. \newline
        A utilização do TDD garante segurança na hora do desenvolvimento. Como
        confiamos nos testes que desenvolvemos, se realizarmos alguma alteração
        no código, conseguimos verificar nos testes se nossa alteração gerou um
        erro em alguma funcionalidade e, com base nos testes, conseguimos tratar
        este erro antes de liberar esta alteração para os membros do equipe, já
        que estamos utilizando um sistema versionador de código. Os testes também
        conseguem demonstrar como o código testado funciona \cite{martin2007agile}:
        como os requisitos estão sendo representados através dos testes que realizamos,
        a descrição de como funciona a nossa funcionalidade, quais são os parâmetros
        que ela recebe e qual é o seu objetivo, podem ser compreendidos através dos
        testes montados. Com eles, conseguimos enxergar o cenário que a funcionalidade
        está sendo executada e o que deve acontecer com o sistema após a sua execução.
        \newline
        A prática de TDD requer que os testes sejam executados de forma automatizada.
        Desta forma conseguimos que todas as nossas funcionalidades sejam testadas
        e os erros de cada alteração que realizamos no código, caso aconteçam,
        sejam verificados e corrigidos para rodar novamente o \textit{pipeline}
        e verificar se a nossa implementação está passando nos testes após o ajuste
        \cite{ContinuousDelivery}. Cada classe desenvolvida deve ter uma classe
        de teste em que devemos verificar se o objetivo da classe está sendo atingido
        e os requisitos que estamos trabalhando na classe estão sendo cumpridos.
        Para aumentar a quantidade de testes automatizados podemos criar testes
        que executem automaticamente operações no sistema, simulando cliques na
        tela, desta forma podemos construir \textit{scripts} de testes, de forma
        a validar que o que foi desenvolvido está funcionando da maneira esperada,
        simulando a utilização do usuário. Idealmente, os testes devem estar inseridos
        dentro de alguma fase do \textit{pipeline}, para que antes de qualquer
        \textit{deploy}, as verificações mencionadas sejam verificadas. \newline

      \subsection{Sustentabilidade}
        Agora que implementamos o nosso sistema com qualidade em produção, devemos
        nos assegurar que o nosso sistema continue em funcionamento. Como a utilização
        de um sistema pode variar muito dependendo da época em que ele está sendo
        utilizado, muitas vezes não conseguimos prever quais os impactos que uma
        grande quantidade de acessos pode causar, um ataque a segurança do nosso
        sistema, um servidor que se encontrou fora do ar, diversos imprevistos podem
        ocorrer em produção e que podem impactar negativamente o nosso produto. \newline
        Para nos prepararmos para esses imprevistos, e conseguirmos sustentar o
        nosso sistema devemos nos preocupar com que o nosso ele tenha escalabilidade,
        esteja disponível, se mantenha seguro e que sua manutenção seja simples
        de ser realizada. Para isso devemos investir na sustentação do nosso sistema.
        Através da utilização de métricas, conseguimos verificar como o nosso sistema
        está funcionando e onde está apresentando sinais de problemas \cite{
        TheDevOpsHandbook}, desta forma conseguimos nos preparar para problemas
        que possam ocorrer em produção. Os problemas que ocorrerem em produção e
        os dados que coletarmos no nosso sistema, deverão servir como meio de
        aprendizagem para que possamos descobrir como nos preparar para eventuais
        ocorrências em produção e manter nosso sistema disponível e confiável \cite{
        SiteReliabilityEngineering}.
