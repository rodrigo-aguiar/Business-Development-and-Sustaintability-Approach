\part{Fundamentos conceituais}
  \chapter{Análise geral sobre desenvolvimento de \textit{software}}

    \section{Práticas ágeis}
      Na história, no começo no desenvolvimento de \textit{software}, a metodologia
      mais utilizada para realizar a produção de \textit{software} era a \textit{waterfall},
      que consiste em um desenvolvimento faseado, primeiro específicamos todo o sistema,
      depois desenvolvemos, depois testamos e por fim colocamos em produção. Porém,
      muitas vezes quando terminávamos de especificar o sistema todo, as necessidades
      do negócio havia mudado, o que gerava mudança no desenho do sistema e fazia
      o time voltar para a fase de especificação. O que tambem acontecia, era que
      durante o desenvolvimento, as necessidade do negócio mudavam novamente e
      o time precisava fazer a escolha de voltar para a fase de desenho ou continuar
      o desenvolvimento da forma que foi desenhado. Esta decisão não era fácil e
      gerava muitos conflitos entre a área de negócio e o time de desenvolvimento
      pois muito do que foi realizado teria que ser refeito. \newline
      Para solucionar este problema em fevereiro de 2001 foi publicado o manifesto
      ágil, um conjunto de valores e princípios para auxiliar o desenvolvimento de
      \textit{software} seja realizada de forma mais rápida e adaptativa as
      necessidades do negócio, onde era mais enfatizado a necessidade de comunicação
      entre as pessoas do que o planejamento excessivo, a adaptação a mudanças ao
      invés de soluções definitivas e a priorização da realização do produto ao
      invés de seguir processos e ferramentas.

      \begin{center}
        \textbf{Manifesto para Desenvolvimento Ágil de Software}

        Estamos descobrindo maneiras melhores de desenvolver software, fazendo-o
        nós mesmos e ajudando outros a fazerem o mesmo. Através deste trabalho,
        passamos a valorizar: \newline
        \textbf{Indivíduos e interações} mais que processos e ferramentas \newline
        \textbf{Software em funcionamento} mais que documentação abrangente \newline
        \textbf{Colaboração com o cliente} mais que negociação de contratos \newline
        \textbf{Responder a mudanças} mais que seguir um plano \newline
        Ou seja, mesmo havendo valor nos itens à direita, valorizamos mais os
        itens à esquerda.

        \begin{tabular}{c c c}
          Kent Beck & James Grenning & Robert C. Martin \\
          Mike Beedle & Jim Highsmith & Steve Mellor \\
          Arie van Bennekum & Andrew Hunt & Ken Schwaber \\
          Alistair Cockburn & Ron Jeffries & Jeff Sutherland \\
          Ward Cunningham & Jon Kern & Dave Thomas \\
          Martin Fowler & Brian Marick \\
        \end{tabular}

        \cite{manifestoAgile}

      \end{center}

      Conforme apresentado no Agile Principles, Patterns, and Practices in C\#
      \cite{martin2007agile}, o manifesto ágil valoriza as pessoas que estão
      no desenvolvendo o \textit{sofware}, mais do que processos e contratos,
      não é ncessário ter desenvolvedores extramamente experientes, mas sim
      um time que esteja unido e que seja multidisciplinar, ou seja, que alem de
      desenvolvedores, é necessário manter as pessoas com entendimento sobre o
      negócio para que durante o desenvolvimento, dúvidas e mudanças no negócio
      possam ser explicadas e o time cpmseguir desenvolver um produto de maior
      valor para o negócio. É importante notar, que sob a visão ágil, mudanças
      nos requisitos do projeto são bem vindas pois, o produto que está sendo
      construído pe mais importante do que documentos e processos, é mais
      importante atender as necessidades do negócio do que seguir um planejamento,
      ou seja, embora seja importante planejar, precisamos estar preparados para
      mudanças e imprevistos. Desta forma, os desenvolvedores devem construir
      uma sistema que seja fácil de alterar, é necessário que o que está sendo
      desenvolvido, possa ser alterado conforme as necessidades do negócio. \newline
      Com esse valores e princípios, a produção de \textit{software} se tornou
      mais rápida, o que antes demorava anos para ser produzido começou a ser
      realizado em meses pois, o sistema começou a ser desenhado conforme era
      desenvolvido, os desenvolvedores não desenhavam promeiro todo o sistema
      para só depois começar o desenvolvimento, ao invés, eles já começavam a
      desenvolver sem se preocupar muito com a documentação, e a cada mudança
      que ocorria, não é mais necessário desenhar novamente toda a solução, mas
      sim adaptar o que já foi desenvolvido. \newline
      As metodologias advindas das práticas ageis, promovem o conceito de iterações,
      que consiste em um curto espaço de tempo, normalmente semanas, seja produzido
      \textit{software} em produção, para que desta forma consigamos avaliar os
      impactos que o \textit{software} produz para o negócio e em menos de um
      mês seja possível já mensurar o valor que está sendo produzido para o
      negócio. Como o tempo de desenvolvimento é curto, está previsto que vamos
      errar diversas, muitas vezes as soluções que vamos produzir não vai atender
      todas as necessidades do negócio porém, é através do aprendizado destes erros
      que vamos aprendendo as necessidades do negócio e corrigindo o sistema para
      que ele se torne cada vez mais aderente as necessidades, temos que errar o
      mais rápido possível para que possamos aprender com os nossos erros e acertar
      o mais rápido possível.

    \section{\textit{DevOps}}
      Através dos valores e das práticas ágeis e com o surgimento de novas
      tecnologias, mais práticas foram sendo criadas e novas ferramentas surgiram
      para auxiliar o desenvolvimento de \textit{software}, já com a perspectiva
      ágil. Desta forma surgiu o conjunto de práticas denonimado \textit{DevOps}, que
      consiste auxiliar o time de desenvolvimento e de operação, trabalharem de
      forma mais unida. Conforme o tempo de desenvolvimento dos produtos foi
      encurtando, surgiu a necessidade de realizar o \textit{deploy} em produção
      diversas vezes ao longo do dia pois, como trabalhamos de forma adaptativa
      e valorizamos o produto sendo estregue de forma rápida, para gerar valor
      o mais rápido possível, a infraestrutura montada e o time de operações,
      precisam estar preparados. Embora seja importante realizar entregas rápidas,
      ainda é necessário avaliar a qualidade do que está sendo entregue, para
      evitar erros em produção e não tornar algo que deveria gerar valor se
      tornar um problema para o negócio. \newline
      As práticas de \textit{DevOps}, nos fornece um conjunto de ações que podemos
      realizar para que consigamos evitar que os erros aconteçam e que caso venham
      a acontecerm, teremos as informações necessárias para avaliar o ocorrido e
      resolver da melhor forma o mais rápido possível. \newline
      Entre as princípais práticas está o de \textit{code review}, que consiste
      que antes de um \textit{deploy} ocorrer, o código que foi produzido será
      verificado em busca de falhas, má qualidade de código, erros de lógica,
      brechas de segurança, ou seja, qualquer coisa que possa causar algum erro
      em produção. Outra prática importante é a de estratégias de ambientes,
      quando estamos produzindo um \textit{software} não é aconselhável realizar
      alterações direto em produção, sem antes ela ter sido testada e o código
      ter sido avaliado, desta forma, é importante ter um ambientes separado de
      produção, onde seja possível os desenvolvedores realizar testes das
      funcionalidades que estão sendo produzidas, para assegurar a qualidade do
      que foi desenvolvido e conseguir avaliar se está aderente a necessidade do
      negócio. \newline
      O \textit{DevOps} apresenta que o processo de desenvolvimento de levar em
      considerção todos os processos de verificação de qualidade e efetivamente
      o \textit{deploy} em produção. A equipe de operação, que é responsável por
      realizar o \textit{deploy}, faz parte do time de desenvolvimento e o tempo
      que será utilizado para realizar o \textit{deploy}, levando em conta
      \textit{code review}, alterações na infraestrutura do sistema, o tempo do
      \textit{deploy}, análises de segurança, entre outro qualquer processo que
      deva ser realizado para assegurar um \textit{deploy} com qualidade, verificando
      a presença de possíveis \textit{bugs}, deve ser levado em conta durante o
      planejamento da iteração, e caso venha a aparecer algum ponto que deve ser
      ajustado pelos desenvolvedores, eles devem ser notificados o mais rápido o
      possível, para que seja realizado o ajuste e o \textit{deploy} possa acontecer.

    \section{CI / CD}
      Para otimizar esta comunicação, e facilitar adicionar estes processos de
      verificação do código para possibilitar o \textit{deploy}, foi criado os
      conceitos de CI (\textit{continuous integration}) e CD
      (\textit{continuous delivery}), que consiste [...]

  \chapter{Qualidade de software}

    \section{Requisitos funcionais}

    \section{Requisitos não funcionais}

    \section{Requisitos inversos}

    \section{Escalabilidade}

  \chapter{Testes no desenvolvimento de \textit{software}}

    \section{Testes unitarios}

    \section{Testes de regrssão}

    \section{Testes de fumaça}

    \section{Testes de integração}

    \section{Testes automatizados}
