\chapter{Proposta}
  \capepigrafe[0.5\textwidth]{``Não tente se tornar uma pessoa de sucesso,
  prefira tentar se tornar uma pessoa de valor.''}{Albert Einstein}

  \section{Como criar produtos com qualidade?}
    Quando produzimos um \textit{software}, focamos sempre nas funcionalidades que
    o sistema deve conter, pretendemos entregar o mais rápido possível e respeitar
    as datas alinhadas com o negócio. Porem, na pressa de colocar as funcionalidades
    em produção, desconsideramos os requisitos não-funcionais e por mais que a
    funcionalidade tenha sido entregue conforme o específicado, o usuário não
    consegue utiliza-la, consequentemente, está entrega não está gerando valor. \newline
    Mas como assim, os requisitos foram atendidos mas o usuário não consegue utilizar?
    Como ignoramos os requisitos não-funcionais, o usuário pode estar sofrendo por
    diversos problemas, por exemplo, um \textit{layout} confuso, um desempenho baixo,
    o sistema está forçando ele a realizar uma tarefa que ele não está acostumado a
    fazer, ou pelo menos não no momento em que apareceu no sistema. Como entregamos
    vizando a data da entrega e nãp o valor para o usuário, aceitamos o risco de
    entregar algo que o usuário não vê valor, e por diversas vezes cometemos erros.
    Ao testar uma funcionalidade, não costumamos ter a visão total do negócio,
    focamos na funcionalidade, então questões como tempo de resposta, ordem lógica
    no processo, localização das informações, nos passam despercebidos, não estamos
    utilizando o sistema diariamente para entender estas questões sozinhos, e o
    negócio muda frequentemente, então quando finalmente entendemos a realidade do
    usuário, ela já mudou, e voltamos a estar sucetiveis ao erro. \newline
    Como não atendemos a expectativa do usuário, e a funcionalidade não está gerando
    valor, implementamos diversas melhorias, todas o mais rápido possível, o que
    acaba gerando \textit{bugs}, como erramos uma vez, precisamos corrigir o erro
    o mais rápido possível, o que nos faz ignorar novamente requisitos não-funcionais
    e, por sua vez, ignoramos questões de arquitetura de \textit{software} o que
    acaba deixando o sistema mais complexo, dificulta a sua manutenção, e deixa o
    sistema mais sucetível a erros. \newline
    Não queremos mais cometer estes erro, queremos cria o produto perfeito que
    será totalmente aderente ao usuário e que irá gerar valor para o negócio. \newline
    O primeiro passo para gerar um produto de qualidade, que auxilia a empresa a
    concluir os seus objetivos e que traga valor ao negócio e entender que não
    existe produto perfeito. Somos humanos, erramos constantemente, realizamos
    escolhas erradas, nos equivocamos. Para realizar uma entrega de pouco risco
    e extremamente acertiva, é necessário muito tempo de planejamento e de estudo,
    como por exemplo a construção de uma avião, ou de um prédio, ou de um
    equipamento hospitalar, este tipo de produto não pode ter falhas, pois caso
    tenha, é um risco para a vida das pessoas. Porem, nosso negócio é mais
    dinâmico, as necessidades muda com mais frequência do que as necessidades de
    um avião, se utilizarmos tanto tempo para planejar como podemos seguir as
    mudanças do negócio?

    \subsection{Assumindo riscos}
    Para conseguirmos acompanhar o negócio e gerar valor para a empresa, evemos
    escolher os riscos que vamos enfrentar. Mesmo um avião encara riscos em seu
    projeto, por isso que é implementado diversas redundancias nas funcionalidades
    mais importantes. Devemos pensar de forma semelhante, qual o objetivo do nosso
    projeto? Qual é sua funcionalidade mais importante? Quais são os riscos que
    vamos encarar? \newline
    Com base nisso, podemos focar a maior parte de nosso tempo no \textit{core}
    do produto, descobrindo o objetivo do sistema, podemos alinhar com os objetivos
    da empresa, assim gerando o retorno esperado. Para realizar o levantamento
    da funcionalidade é importante considerar o nome da funcionalidade, a sua
    importância, os objetivos em que ela vai nos auxiliar a alcançar, os riscos
    que vamos enfrentar com ela e as dependências para disponibilizar a
    funcionalidade ao usuário.

    \begin{table}[h!]
      \centering
      \begin{tabular}{|c|p{10cm}|}
        \hline
        \textbf{Funcionalidade} &
        Nome da funcionalidade que será desenvolvida. \\ \hline
        \textbf{Importância} &
        A importância que a funcionalidade representa para o negócio, separa entre
        alta, média e baixa. \\ \hline
        \textbf{Objetivos} &
        Os objetivos que está funcionalidade irá auxiliar à alcançar. \\ \hline
        \textbf{Riscos} &
        Quais os riscos que essa funcionalidade pode gerar para o negócio. \\ \hline
        \textbf{Dependências} &
        Quais são as dependências para a utilização e desenvolvimento dessa
        funcionalidade. \\ \hline
      \end{tabular}
      \caption{Definição da tabela de funcionalidaes}
      \label{Tabela:1}
    \end{table}

    Toda funcionalidade deve ter um nome que represente o que será desenvolvido
    pois, durante o desenvolvimento, é através do nome que os desenvolvedores
    vão se comunicar, sem um nome claro, que represente bem o negócio, os
    desenvolvedores não vão conseguir se comunicar de forma clara para entender
    as necessidades do negócio. Termos e nomeclaturas devem ser estabelecidas,
    para que o usuário entenda como o sistema deve ser utilizado e para os
    desenvolvedores compreenderem como o sistema será utilizado. Através desta
    comunicação, o time de desenvolvimento consegue extrair os requisitos de forma
    mais fácil e formular um padrão de qualidade. \newline
    É necessário colocar a importância da funcionalidade que será desenvolvida pois,
    com base nela podemos entender a ordem que vamos iniciar o desenvolvimento
    e quais os riscos podemos enfrentar e através da importância podemos definir
    o rigor dos testes que devemos realizar no desenvolvimento da funcionalidade
    e o padrão de qualidade para cada nível de importância. Embora colocamos
    somente três níveis de importância (alto, médio e baixo), cada negócio pode
    ter mais níveis, porém recomendamos não exagerar e passar de cinco níveis,
    pois o principal objetivo desta classificação é forçar escolhermos quais
    funcionalidades são realmente mais importantes, se colocarmos muitos níveis,
    a menor classificação será "alta". \newline
    As funcionalidades que vamos desenvolver tem que estar relacinada com um
    objetivo da empresa, pois é isso que dá propósito para ela, e a nossa base
    para verificar se ela está gerando valor. Com base no uso da funcionalidade,
    podemos verificar se o objetivo está sendo alcançado, e com base nesse
    retorno, podemos decidir quais serão os nossos próximos passos. \newline
    Quando pensamos em uma funcionalidade, devemos sempre considerar os riscos
    para o negócio, a utilização de um sistema implica em com a operação da
    empresa trabalha, e toda mudança gera riscos para a operação. Tudo que é
    novo, deve ser ensinado, por mais que a funcionalidade reflita exatamente o
    que os usuários já faziam, eles vão começar a eecutar essas atividades em
    um outro lugar, o que no começo pode gerar dúvidas e frustrações. Devemos
    sempre levantar os riscos com base na perspectiva do usuário, pois assim
    podemos entender quais serão as reações delese e quais estratégias devemos
    utilizar para engajar o uso da ferramenta e buscar por melhorias. \newline
    Com base nos objetivos, nas funcionlidades e nos riscos, podemos mapear as
    pendências das funcionalidades, onde conseguimos traçar quais os passos que
    devemos tomar para iniciar o desenvolvimento até disponibilizar a funcionalidade
    para o usuário. \newline
    Para o levantamento dessas informações recomendamos utilizar o formato de
    tabela. A tabela possibilita consolidar de forma clara cada tópico e sempre
    que a empresa tiver um novo objetivo, é mais fácil analisar o que já foi listado,
    possibilitando ver se o sistema se encaixa com essa nova necessidade ou se será
    necessário uma nova funcionalidade, ou se será necessário um novo produto,
    ou uma reformulação do negócio. Quando listamos tudo que estamos fazendo
    junto com o que o negócio pretende alcançar, podemos tomar decisões mais
    acertivas, entender dependências e otimizar o valor que será desenvolvido.
    Muitas vezes focamos em desenvolver uma funcionalidade que não irá gerar
    muito valor para o negócio, o que nos faz aceitar um risco que não precisamos
    no momento, toda funcionalidade gera riscos, então é melhor nos concentrarmos
    somente naqueles que podem gerar mais valor. \newline
    Uma vez definido nossas funcionalidade, sua importância e os objetivos que
    pretendemos com elas, podemos escolher quais vamos implementar primeiro e
    quais riscos nós vamos enfrentar. \newline
    Para exemplificar a utilização da tabela proposta vamos considerar uma
    empresa que possuí vendores que entram em contato com outras empresas para
    negociar a venda de computadores. Estes computadores podem ser customizados
    pelo cliente, solicitando maior espaço de memória, se deseja um computador
    com \textit{SSD} ou \textit{HD}, qual o processador, entre outras ecolhas.
    Embora já existam computadores pré-montados, eles podem ser customizados
    somente aproveitando a base do produto. \newline
    Hoje a venda é feita sem a utilização de um sistema, a especificação que o
    cliente deseja é feita de forma indivídual por cada vendedor e controlado por
    meio de planilhas, onde cada vendedor possui a sua forma de organizar. Somente
    o pedido final é colocado em um sistema de controle de pedidos, para emitir a
    nota. A comunicação entre os vendedores e a área técnica é feita por meio de
    telefone e email, os produtos base e os produtos vendidos são compartilhados
    através de planilhas enviadas por email. Como cada vendedor negocia de uma
    forma, nã há Padronização nas planilhas, o que acaba gera confusões na área
    técnica na hora da montagem dos produtos. Outro ponto a destacar é que a área
    administrativa não tem visibilidade de como as negociações estão sendo realizadas
    o que dificulta o entendimento do negócio como um todo e a cração de
    estratégias. \newline
    Com base nessas necessidades, a empresa decidiu começar um projeto para a
    criação de um sistema para os vendedores realizar as suas negociações.
    Onde o produto base seria disponibilizado na ferramenta e os pedidos finais
    seriam montados conforme a negociação avança. Outra necessidade, é que a
    negociação deve ser faseada, para que seja possível rastrear os passos do
    vendedor e ter uma melhor visão do negócio. \newline
    Neste cenário, o primeiro passo é levantar as princípais funcionalidades para
    atender o negócio, já colocando a sua importância e os objetivos que pretendemos
    alcançar.

    \begin{table}[h!]
      \centering
      \begin{tabular}{|c|c|p{8cm}|}
        \hline
        \textbf{Funcionalidade} &
        \textbf{Importância}  &
        \textbf{Objetivos} \\ \hline
        %Funcionalidade
        Cadastro de produtos &
        %Importância
        Alta &
        %Objetivos
        Padronização na criação de produtos; \newline
        Reduzir o cadastro incorreto de produtos.
        \\ \hline
        %Funcionalidade
        Venda de produtos &
        %Importância
        Alta &
        %Objetivos
        Venda faseada do produto; \newline
        Melhor rastreabilidade da venda; \newline
        Padronização da negociação.
        \\ \hline
        %Funcionalidade
        \textit{Chat} interno &
        %Importância
        Baixa &
        %Objetivos
        Otimização da comunicação entre a área técnica e os vendedores.
        \\ \hline
      \end{tabular}
      \caption{Exemplo de levantamento de funcionalidades}
      \label{Tabela:2}
    \end{table}

    Na tabela dois, representamos três funcionalidades, o cadastro de produtos,
    a venda de produtos, e o \textit{chat} interno. Podemos notar que a maior necessidade
    da empresa é a de padronizar a venda do produto, onde a negociação deve ser
    feita de forma mais uniforme e que facilite os vendedores a venderem produtos
    mais aderentes com o que a área técnica consegue produzir. Com base nestas
    necessidades, foi levantado as funcionalidades de cadastro de produtos e
    venda de produtos, onde o cadastro está relacionada com a padrinização da
    criação do produto e a venda está relacionada com a padronização da venda.
    Porem, a funcionalidade de \textit{chat} interno, não é uma prioridade para
    o negócio, as áreas estão se comunicando, o problema é que cada vendedor
    trabalha de um jeito, o que dificulta as decisões da área técnica. Embora um
    dos objetivos seja melhorar a comunicação entre as áreas, este não é o maior
    dos problemas que precisamos resolver, então os riscos que desta funcionalidade
    não precisam ser corridos até o desenvolvimento das outras duas. \newline
    Outro ponto de destaque é a identificação das dependências, quando mapeamos
    as dependências, conseguimos ver qual funcionalidade precisamos realizar
    primeiro, como listado no exemplo, não é possível vender os produtos sem antes
    cadastrá-los. Mas alem da dependências entre funcionalidades, podemos listar
    o que precisamos fazer para disponibilizar a funcionalidade para o usuário,
    como apresentado no exemplo, todas as funcionalidades precisam de treinamento,
    como o sistema é novo e cada vendedor realiza as negociações do seu jeito,
    é importante explicar como foi montado o processo de vendas e apresentar
    como realizar as atividades no sistema. \newline
    Cada funcionalidade tem os seus riscos, porem é através de sua importância que
    podemos criar estratégias para enfrentar estes riscos e marcar estas ações
    que definimos em nossa estratégia como dependências para liberar o uso da
    funcionalidade, é importante atrelar estas ações como dependências, pois desta
    forma os usuários não são surprendidos, e é através dos treinamentos e das
    apresentações que podemos pegar o \textit{feedback} dos usuário e começar a
    mapear novas funcionalidades e melhorias. Podemos ver o mapeamento dos riscos
    e das dependências das funcionalidades exemplo na tabela três.\newline

    \begin{table}[h!]
      \centering
      \begin{tabular}{|c|p{4cm}|p{6cm}|}
        \hline
        \textbf{Funcionalidade} &
        \textbf{Riscos}  &
        \textbf{Dependências}
        \\ \hline
        %Funcionalidade
        Cadastro de produtos &
        %Riscos
        Processo de cadastro de produtos mais demorado; \newline
        Criação de produtos menos flexível. &
        %Dependências
        Treinamento para cadastrar os produtos no sistema;\newline
        Explicação das regras utilizadas para o cadastro dos produtos.
        \\ \hline
        %Funcionalidade
        Venda de produtos &
        %Riscos
        Venda menos flexível; \newline
        Mudança na forma de negociação dos vendedores; &
        %Dependências
        Funcionalidade de cadastro de produtos; \newline
        Treinamento para realizar vendas no sistema; \newline
        Explicação das fases na venda.
        \\ \hline
        %Funcionalidade
        \textit{Chat} para clientes &
        %Riscos
        Não adoção do \textit{chat} como princípal meio de comunicação; \newline
        Utilização do \textit{chat} para assuntos sem relação ao trabalho; &
        %Dependências
        Treinamento para a utilização do \textit{chat}; \newline
        Treinamento sobre como se comunicar por \textit{chat}.
        \\ \hline
      \end{tabular}
      \caption{Exemplo de levantamento dos riscos e dependências por funcionalidade}
      \label{Tabela:3}
    \end{table}

    \subsection{Entendendo os requisitos de uma funcionalidade}
    \subsection{Arquitetura evolutiva}
    \subsection{Estruturando o fluxo de entrega}
    \subsection{Automatizando testes de qualidade}

  \section{Como adquirir escalabilidade?}
    \subsection{Descobrindo os limites do sistema}
    \subsection{Desenvolvendo estratégias de escalabilidade}

  \section{Como manter o sistema disponível?}
    \subsection{Utilizando o negócio para definir disponibilidade}
    \subsection{Analisando o desempenho do sistema}
    \subsection{Falando com os usuários}
    \subsection{Utilizando métricas para assegurar a disponibilidade}
    \subsection{Aplicando testes automatizados}

  \section{Como identificar falhas?}
    \subsection{Sentindo o cheiro do produto}
    \subsection{Utilizando o usuário para identificar falhas}
    \subsection{Solucionando falhas}
    \subsection{Aprendendo com os erros}
    \subsection{Testes automatizados como ferramenta de aprendizado}
    \subsection{Divulgando as descobertas de forma global}
