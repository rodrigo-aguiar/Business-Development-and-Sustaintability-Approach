\chapter{Introdução}
O ambiente de desenvolvimento de \textit{software} passou por diversas mudanças
durante o tempo. Assuntos como produtividade, qualidade, adaptação a mudanças e
manutenção em tempo real, assuntos que é pesquisado até os dias de hoje. \newline
Com o nascimento desses temas surgiu novas tarefas para serem realizadas, e
com estas tarefas, novos cargos e papeis. Porem, para realizar estas novas
práticas, muitas vezes é necessário diversas pessoas, possuindo diversos papeis
diferentes, e com isso, a quantidade de membros em um time cresce e fica difícil
manter a gestão. \newline
[Buscar uma pesquisa para confirmar o aumento de pessoas com diminuição da
produtividade] \newline
Com a utilização das práticas de \textit{devops}, ferramentas de automação e com
a aplicação de inteligência artificial, vamos possibilitar que as principais
análises em um produto sejam realizadas de forma automatizadas, possibilitando
correções antes que um problema aconteça e facilitando a análise de impactos sobre
uma nova \textit{feature} e impedir que problemas aconteçam antes de acontecerem. \newline

\section{Motivações}
Buscamos analisar as principais práticas e buscar formas para automatizar
os processos sugeridos por elas, com o intuito de manter os times com tamanhos
pequenos e as praticas mais simples. Com isso, acreditamos que os times serão
mais produtivos e os produtos serão entregues com maior qualidade, garantindo
vantagem competitiva para o negócio. \newline
Nossa maior motivação, é possibilitar que os projetos de grande porte possa
manter sua escalabilidade sem perder a produtividade. Garantir que os times
sigam práticas para garantir a qualidade do produto que estão entregando e que
um projeto que se inicie pequeno possa evoluir de forma natural. \newline
[Buscar pesquisa sobre projetos que cresceram mal]
