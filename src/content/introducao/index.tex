\chapter{Introdução}
O ambiente de desenvolvimento de \textit{software} passou por diversas mudanças
durante o tempo. Assuntos como produtividade, qualidade, adaptação a mudanças e
manutenção em tempo real, assuntos que é pesquisado até os dias de hoje. \newline
Com o nascimento desses temas surgiu novas tarefas para serem realizadas, e
com estas tarefas, novos cargos e papeis. Porem, para realizar estas novas
práticas, muitas vezes é necessário diversas pessoas, possuindo diversos papeis
diferentes, e com isso, a quantidade de membros em um time cresce, com membros
de papeis variados tornando difícil manter a gestão e a comunicação entre o
time. \newline
[Buscar uma pesquisa para confirmar o aumento de pessoas com diminuição da
produtividade] \newline
Com a utilização das práticas de \textit{devops}, ferramentas de automação e com
a aplicação de inteligência artificial, buscamos possibilitar que a comunicação
entre os membros do time sejam realizadas de forma mais fácil e mais rápida,
sugerindo padrões a serem tomados pelo time e automatizando algumas tarefas. \newline

\section{Motivações}
Nossa maior motivação, é possibilitar que os projetos de grande porte possa
manter sua escalabilidade sem perder a produtividade. Garantir que os times
sigam práticas para garantir a qualidade do produto que estão entregando e que
um projeto que se inicie pequeno possa evoluir de forma natural. \newline
[Buscar pesquisa sobre projetos que cresceram mal] \newline
Através de nossas análises nas principais práticas de desenvovimento e gestão
de produtos de \textit{software}, apresentamos formas para automatizar tarefas
repetitivas com o intuito de manter os times concentrados na melhoria do produto
e tornar a execução dessas práticas mais simples. Com isso, acreditamos que
o desenvolvimento dos produtos serão mais produtivos e serão entregues com maior
qualidade. \newline
