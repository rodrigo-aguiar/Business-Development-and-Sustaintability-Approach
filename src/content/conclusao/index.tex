\part{Conclusão}
  \chapter{Resultados em relação ao objetivo}
    Cada vez mais, os negócios estão passando por transformações, a reestruturação
    dos modelos de negócio estão cada vez mais comuns, para manter um negócio é
    necessário sempre rever a forma de trabalho da empresa e entender se ela está
    aderente com as movimentações do mercado. \newline
    Com esta abordagem, nosso objetivo foi o de auxiliar o desenvolvimento de novos
    produtos, focando na qualidade do que está sendo desenvolvido e apresentando
    formas de auxiliar a empresa a evoluir o sistema e o negócio. Em relação
    ao que foi proposto, acreditamos que a utilização desta abordagem facilite as
    equipes nas tomadas de decisões sobre como o produto deve ser desenvolvido,
    como garantir que os requisitos estejam sendo assegurados e que os riscos em
    cada implementação estejam evidentes para todos os envolvidos no projeto.\newline
    Nós também acreditamos, que o sistema deve trazer novas visões do negócio, é
    através dos dados que são gerados com a utilização dos usuários que conseguimos
    captar novas visões de como o negócio funciona e desenvolver estratégias
    para adquirir maior vantagem competitiva. Com essas novas estratégias, devemos
    mapear novos requisitos e com eles, aprimorar nosso produto. \newline
    Com a vinda de novos requisitos e a mudança de requisitos já mapeados, conseguimos
    avaliar se o nosso produto continua aderente ou se ele já cumpriu o seu objetivo
    e devemos começar o desenvolvimento de um novo. Lembrando, como nosso produto
    está escalável, e nossas funcionalidades separadas em sistemas separados, nosso
    produto agora não é somente um sistema, mas sim sistemas, e por mais que vamos
    ter que redefinir a sua estrutura e repensar no seu objetivo, as funcionalidades
    que ainda estão aderentes, deverão continuar no nosso próximo produto, desta
    forma conseguimos maior agilidade no desenvolvimento e conseguimos focar
    diretamente na nova perspectiva da empresa, mantendo aquilo que continua
    consolidado.

  \chapter{Trabalhos futuros}
    Como futuro trabalho, podemos evoluir esta abordagem como um método focando
    nas questões de arquitetura e em requisitos de qualidade, aonde vamos especificar,
    mais como garantir a escalabilidade e apresentando mais fatores de qualidade
    além de disponibilidade de detecção de falhas. Podemos evoluir os processos de
    desenvolvimento apresentados, com mais detalhes e apresentando a abordagem sendo
    utilizada durante projetos reais.
