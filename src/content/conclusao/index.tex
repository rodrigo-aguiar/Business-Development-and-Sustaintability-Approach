\part{Conclusão}
  \chapter{Resultados em relação ao objetivo}
    Nosso objetivo é o de produzir \textit{software} escalável e de qualidade,
    onde as métricas utilizadas em produção são utilizadas para adquirir maior
    entendimento do negócio e supervisionar a saúde do sistema. \newline
    Durante o trabalho foi apresentado práticas para serem executadas com o
    intuíto de desenvolver a escalabilidade do sistema, utilizando os problemas
    enfrentados em produção para aprender como melhorar o sistema. Para conseguirmos
    nos prevenir aos erros, específicamos um ambiente que deve ser utilizado para
    simular cenários caóticos de produção, onde, forçamos o sistema no seu máximo
    com o intuíto de provocar problemas que possam ocorrer em produção, desta
    forma, conseguimos nos antecipar a esses cenários e adicionar novas métricas
    para supervisionar o sistema nos antecipar, gerando um plano de ação caso
    aconteça. \newline
    Apresentamos um fluxo de entrega, orientado a testes contínuos verificando
    se uma funcionalidade continua funcionando sempre que é realizada alguma
    alteração no sistema, verificando surante o fluxo de desenvolvimento para
    tentar impedir que aconteça erros em produção. Descrevemos como específicar
    as funcionalidades e os requisitos baseados em riscos e o que eles implicam
    na realização dos testes automatizados e no monitoramento de produção. \newline
    Foi demonstrado como utilizar o desenvolvimento que produzimos como ferramenta de
    aprendizado, onde, é possível através do código gerado, dos testes automatizados
    das métricas criadas e das ferramentas utilizadas, utilizar as informações
    contidas em cada uma dessas partes uma forma de descrever o sistema e o
    negócio, podendo identificar se os objetivos do negócio estão sendo cumpridos
    e uma vez que os objetivo princípal do produto esteja completo, conseguimos
    desenvolver um produto novo se baseando no aprendizado que obtivemos com este
    e possibilitando utilizar novamente funcionalidades que continuam aderentes
    com o negócio.

  \chapter{Trabalhos futuros}
    Como futuro trabalho, podemos evoluir esta abordagem como um método focando
    nas questões de arquitetura e em requisitos de qualidade, aonde vamos especificar,
    com mais detalhes como garantir a escalabilidade e detalhando os fatores de
    qualidade como segurança, disponibilidade, entre outras. Podemos evoluir os
    processos de desenvolvimento especificados com mais detalhes apresentando a
    abordagem sendo utilizada em projetos reais.
