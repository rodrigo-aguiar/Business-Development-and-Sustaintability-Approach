\part{Conclusão}
  \chapter{Resultados em relação ao objetivo}
    Nosso objetivo foi que com a utilização desta abordagem possibilite que o
    desenvolvimento do produto seja realizado com qualidade e com escalabilidade,
    onde as estratégias de melhoria sejam tomadas com base nas mudanças no mercado
    e na sustentação realizada em produção, com base nas métricas adquiridas e na
    forma que o usuário utiliza o sistema, através da analise desses dados, seja
    possível se preparar para problemas, comprender as necessidades dos usuários
    e direcionar as equipes nas tomadas de decisões sobre como o produto deve ser
    desenvolvido, como garantir que os requisitos estejam sendo assegurados e que
    os riscos em cada implementação estejam evidentes para todos os envolvidos no
    projeto.\newline
    O sistema deve trazer novas visões do negócio, e é através dos dados que são
    gerados que conseguimos captar como o negócio funciona e desenvolver estratégias
    para adquirir maior vantagem competitiva mapeando novas necessidades para o
    nosso produto, fazendo surgir novos requisitos e alterando requisitos
    existentes, onde desta forma, conseguimos avaliar se o nosso produto continua
    aderente ou se ele já cumpriu o seu objetivo assim descobrindo a necessidade
    de começar o desenvolvimento de um novo. \newline
    A escalabilidade de nosso produto antigo permite que nossas funcionalidades
    que foram separadas em sistemas, nosso controle sobre os objetivos de cada
    uma delas, nossa infraestrutura adaptativa e nossa arquitetura evolutiva,
    possibilite que as partes que ainda estão aderentes com os objetivos da empresa
    se mantenham e as partes que não estão mais aderentes sejam trocadas, mantendo
    a mesma infraestrura e focando que o novo produto seja desenvolvido mais
    rapidamente e mantendo os aprendizados do produto anterior e verificando se
    o que vai ser reutilizado, quando implementado novamente, se mantem funcionando
    através dos testes automatizados e do monitoramento em produção.

  \chapter{Trabalhos futuros}
    Como futuro trabalho, podemos evoluir esta abordagem como um método focando
    nas questões de arquitetura e em requisitos de qualidade, aonde vamos especificar,
    com mais detalhes como garantir a escalabilidade e detalhando os fatores de
    qualidade como segurança, disponibilidade, entre outras. Podemos evoluir os
    processos de desenvolvimento especificados com mais detalhes apresentando a
    abordagem sendo utilizada em projetos reais.
