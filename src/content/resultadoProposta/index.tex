\chapter{Resultados da proposta}
  \capepigrafe[0.5\textwidth]{``A verdade só pode ser encontrada em um lugar:
  o código''}{Robert C. Martin}

  \section{Um produto com qualidade}
  Ao desenvolver um produto, muitas vezes focamos apenas em requisitos
  funcionais, nos concentramos em entregar o máximo número de funcionalidades
  para os usuários e esquecemos dos requisitos não-funcionais. Conforme apresentado
  no trabalho Quality Attributes and Service-Oriented Architectures
  \cite{O'BrienQualityAttributes2005}, os requisitos não-funcionais estão diretamente
  ligados com a qualidade de nosso \textit{software}, questões como performance,
  usabilidade, segurança e enre outros, são requisitos que garantem que o nosso
  produto funciona e que as nossas funcionalidades desenvolvidas, estão aderentes
  com o negócio. \newline
  Um produto de qualidade deve estar sempre aderente ao negócio, os usuários devem
  conseguir utilizar o sistema e sentir que as necessidades deles estão sendo
  atendidas, eles devem sentir segurança a utilizar o sistema e as funcionalidades
  devem ter um tempo de resposta razoável, para que eles se sintam a vontade ao
  realizar as suas atividades e para que a experiência do usuário não se torne
  frustante. É importante sabermos que não vamos conseguir atender todos os
  requisitos e que teremos que enfrentar riscos no nosso produto, e conforme
  apresentado no The Architecture Tradeoff Analysis Method \cite{KazmanTheArchitecture1998}
  com base nos riscos que analisamos, devemos buscar remodelar a arquitetura que
  desenvolvemos para o nosso produto para que o nosso produto continue com a
  mesma qualidade e possamos mitigar, resolver ou conviver com esses riscos.
  A experiência do usuário e o valor que o produto está gerando para o negócio,
  está diretamente atrelado a qualidade que cada funcnionaliade foi desenvolvida
  e quais os riscos que estamos convivendo, é importante sempre revalidar com o
  negócio, se o que está desenvolvido continua aderente, se algum risco que
  estamos convivendo deveria ser mitigado e com essas análises, se devemos
  investir em reformular a arquitetura montada, para se adaptar as novas definições,
  desenvolver uma nova funcionlidade, atender mais alguma necessidade do negócio
  ou se devemos começar a desenvolver um novo produto pois, o negócio está
  sofrendo diversas alterações que estão descaracterizando o produto desenvolvido
  e que o objetivo do sistema que contruímos foi cumprido e agora a empresa possui
  um novo objetivo. \newline
  Quando desenvolvemos nosso produto em conjunto com o negócio, conforme apresentado
  no Domain Driven Design \cite{DomainDrivenDesign} conseguimos criar uma linguagem
  que será utilizada para que as áreas de negócio e o time de desenvolvimento
  consigam se comunicar e que desta forma, o time de desenvolvimento consegue
  captar melhor os requisitos das funcionalidades e os usuários conseguem
  compreender mais fácilmente como o sistema funciona. Além deste entendimento
  entre as duas partes, é mais criar um código mais limpo, pois os padrões e os
  nomes dos objetos e das entidades serão nomeados baseado em um entendimento
  que todos no time adquiriram ao se comunicar com o negócio. Padrões e estilo de
  desenvolvimento bem definidos entre os membros do time de desenvolvimento facilita
  a compreensão e manutenção do código, como apresentado no Clean Code \cite{CleanCode},
  a maior parte de nosso tempo utilizamos lendo o nosso código do que escrevendo
  linhas novas, e se o código estiver representando o negócio, tanto em sua
  funcnionaliade quanto em sua escrita, ele se torna a documentação do projeto,
  demonstrando como cada parte do sistema funciona e quais requisitos ele está
  atendendo. \newline
  Alem de garantir que o nosso sistema está aderente com o negócio, e que seja
  possível adaptar a sua arquitetura conforme as necessidades do negócio precisamos
  garantir que ele continue em funcionamento em todos os tipos de cituações,
  precisamos utilizar de métricas, produzir \textit{logs}, conversar com o usuário,
  precisamos garantir que caso um problema venha a acontecer, consiguimos supervisionar
  a saúde do ambiente e encontrar a raíz do problema de forma rápida, e após a
  sua resolução, aprender com ele e desenvolver novas formas de supervisão, para
  que consigamos verificar se o sistema está apresentando sinais que o erro vai
  voltar a ocorrer. Com o aprendizado que obtivemos com o erro, agora podemos
  elaborar estratégias para nos preparar para futuras situações, seja aumentar
  a quantidade de servidores durante um determinado período, seja particionar
  mais o processamento de alguma funcionalidade do sistema ou seja até mesmo
  desativar por completo uma funcionalidade que não está mais sendo utilizada,
  mas ainda consome recursos computacionais, através do negócio e com as situações
  que enfrentamos dia a dia, podemos traçar estratégias, redefinir a nossa
  arquitetura e adquirir maior consciência sobre o negócio como um todo, pois
  desta forma estamos utilzando o negócio para definir o sistema e o sistema
  para aprender sobre o negócio.
