\begin{resumo}
  Este trabalho apresenta uma abordagem de como realizar o desenvolvimento e
  a sustentação de \textit{software} centrado no negócio. A abordagem apresenta
  práticas que devem ser realizadas com o intuito de produzir um sistema com maior
  qualidade que seja mais aderente ao negócio, onde o negócio é utilizado para
  aprendermos como desenvolver o sistema e assegurar a qualidade do que está
  sendo desenvolvido, e o que foi desenvolvido é utilizado para aprendermos como
  o negócio funciona e como podemos melhorá-lo. É aplicado uma forma de aprendizado
  contínuo, onde o processo de desenvolvimento ocorre não apenas para produzir um
  sistema, para descrever o negócio e aplicar formas para assegurar que o
  produto está aderente. É apresentado como especificar as funcionalidades
  do produto, identificar riscos e descobrir quais ações tomar com os riscos
  identificados, se baseando em métricas e testes realizados no sistema durante
  o processo de desenvolvimento. A abordagem também apresenta como agir quando um
  erro em produção é identificado e o que devemos fazer para solucionar o erro e
  garantir que este ele não ocorra novamente ou que seja solucionado da forma
  rápida e eficaz caso volte a acontecer. \\[3\baselineskip]

  \textbf{Palavras-Chave} -- Qualidade. Disponibilidade. Negócio. Escalabilidade.
  Usuário. Testes. Testes Automatizados. Produto. Sistema. Funcionalidade.
\end{resumo}

\begin{abstract}
  This work presents an approach to develop and sustain software based on the
  business. The approach presents a set of practices that should be realized to
  produce software with more quality and there's be more adherent to the business,
  where the business is used to learn about how to develop the software and assure
  your quality and the software is used to learn how the business work and how
  to improve him. Is applied a form of continuous learning, where the process of
  development occurs don't just produce the software but to describe the business and how
  apply ways to assure the product is adherent to the business. It shows ways
  of how to specify the functionalities of the product, identify the risks and
  find what actions take with the identified risks, based on metrics and tests
  realized on the software in the development process. The approach presents too
  how to act when a problem is identified in production and what we should do to
  resolve and assure the error doesn't occur again, or that he be resolved, more
  effectively and fast if he comes back to occur. \\[3\baselineskip]

  \textbf{Keywords} -- Quality. Disponibility. Business. Scalability.
  User. Tests. Automated Tests. Product. Software. functionality.
\end{abstract}
