\begin{resumo}
  Este trabalho apresenta uma abordagem de como realizar o desenvolvimento e
  a sustentação de \textit{software} centrado no negócio. A abordagem específica
  práticas que devem ser realizadas com o intuito de produzir um sistema com
  escalabilidade onde o negócio é utilizado para aprendermos como desenvolver o sistema
  e assegurar a qualidade do que está sendo desenvolvido e o que foi desenvolvido
  é utilizado para aprendermos como o negócio funciona e como podemos melhorá-lo.
  É aplicado uma forma de aprendizado contínuo, o processo de desenvolvimento ocorre
  não apenas para produzir um sistema, mas para descrever o negócio e aplicar
  formas de assegurar que o produto está aderente. É apresentado como especificar
  as funcionalidades, identificar riscos e descobrir quais ações tomar com os riscos
  identificados se baseando em métricas e testes realizados durante o processo
  de desenvolvimento e de sustentação. A abordagem também apresenta como agir
  quando um erro em produção é identificado e o que devemos fazer para mitigar
  ou solucionar o erro garantindo que ele não ocorra novamente ou que seja solucionado
  de forma mais rápida e eficaz. \\[3\baselineskip]

  \textbf{Palavras-Chave} -- Qualidade. Negócio. Escalabilidade. Usuário. Testes.
  Testes Automatizados. Produto. Sistema. Funcionalidade.
\end{resumo}

\begin{abstract}
  This work presents an approach to develop and sustain software based on the
  business. The approach specifies a set of practices that should be realized to
  produce software with scalability where the business is used to learn about how
  to develop the software and assure your quality and, the software is used to learn
  how the business work and how to improve him. Is applied a form of continuous
  learning, the process of development occurs don't just to produce software but
  to describe the business and how apply ways to assure the product is adherent.
  It shows ways to specify the functionalities, identify the risks and find what
  actions take with the identified risks based on metrics and tests realized on
  the software in the development and sustain processes. The approach presents too
  how to act when a problem is identified in production and what we should do to
  resolve and assure the error doesn't occur again or that he be resolved faster
  and effectively. \\[3\baselineskip]

  \textbf{Keywords} -- Quality. Business. Scalability. User. Tests.
  Automated Tests. Product. Software. functionality.
\end{abstract}
