\chapter{Metodologia de pesquisa}
Para a realização deste trabalho foi utilizado livros, artigos, \textit{papers}
e notas técnicas para a criação de sistemas com qualidade. Procuramos por trabalhos
que descrevescem o que é qualidade de \textit{softwares} e como construir os
produtos com qualidade. Durante a pesquisa foi identificado a immportância dos
requisitos não-funcionais o que nos levou a procurar por trabalhos que os
apresentassem e demonstrassem técnicas para assegurarar que eles fossem
cumpridos. \newline
Foi utilizado como ferramenta de pesquisa o Google Scholar, ResearchGate e o
Software Engenieering Institute da Carnegie Mellon University.
