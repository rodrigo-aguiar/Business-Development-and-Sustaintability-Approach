\documentclass[]{../class/politex}

\usepackage[num, alf]{abntex2cite}
\usepackage[utf8]{inputenc}
\usepackage{amsmath,amsthm,amsfonts,amssymb}
\usepackage{graphicx,cite,enumerate}
\usepackage[brazil]{babel}

\graphicspath{{images/}}

\titulo{Abordagem de desenvolvimento e sustentação de software orientado a negócio}

\autor{Rodrigo Aguiar Ordonis da Silva}
\orientador{Reginaldo Arakaki}

\mba{Transformações Digítais}
\areaConcentracao{3141 – Engenharia de Computação}
\departamento{Laboratório de Arquitetura e Redes de Computadores}

\local{São Paulo}
\data{2020}

\begin{document}
  \capa
  \falsafolhaderosto
  \folhaderosto

  \dedicatoria{A minha mãe e meu pai, por todo o apoio, amor e carinho.}

  \chapter{Introdução}
  \capepigrafe[0.5\textwidth]{``A confiança perdida é difícil de recuperar.
  Ela não cresce como as unhas.''}{Brahms, Johannes}

  Com o passar do tempo, os \textit{softwares} firacam cada vez mais importantes
  na sociedade e os negócios começaram a depender ainda mais deles. Alem dos
  computadores, há diversos outros dispositovos que nos permitem acesso a internet
  como por exemplo os celulares, os \textit{tablets}, \textit{video games},
  as televisões, e entre vários outros dispositivos. O que facilitou o acesso a
  informação e consequntemente, a divulgação da informação. Com os sistemas ficando
  mais importantes, diversos serviços são realizados pela internet, como compras,
  negociações, comunicação, tranferências bancarias, entre outros.\newline
  Devido a isso, questões como quantidade de acesso, tempo de resposta e segurnaça
  da informação começaram a ficar cada vez mais importantes na concepção de um
  \textit{software}. A queda de um sistema por alguns segundos, pode ocasionar
  em diversos problemas para uma empresa como, a perda de milhões de reais,
  a desvalorização da marca e causar uma reputação negativa para a empresa.
  Dependendo do motivo da queda, pode ocasionar o fim do sistema e por consequência
  o fim de um produto para a empresa. Outro ponto a ressaltar é que devemos entender
  se mesmo com o sistema funcionando corretamente, ele realmente está trazendo retornos
  positivos para a empresa? O usuário está gostando do que disponibilizamos para ele?
  Está falando bem ou mal do produto? \newline
  Para se preparar para estas situações inesperadas e manter o \textit{software}
  funcionando, é importante manter o produto escalável. Não podemos controlar a
  quantidade de acessos no sistema, mas podemos controlar o quanto de recursos
  computacionais é disponibilizado para o \textit{software}, em épocas em que os
  acessos aumentam, podemos aumentar a quantidade de processamento, em épocas que
  diminuem, podemos diminuir o processamento, assim controlamos o custo do sistema
  e garantimos sua disponibilidade. Não podemos tambem ter um \textit{feedback}
  de todos os usuário de como está a experiência na utilização, muitas vezes nem
  da maioria deles, mas podemos analisar as ações dos usuários para entender
  como está sendo a sua experiência. \newline
  Outro ponto que não deve ser esquecido, é de que problemas vão acontecer,
  situações inesperadas que não vamos saber lidar no momento, brechas na segurança,
  um \textit{bug} no sistema, usuários perdidos na navegação, entre outros casos.
  Nestes casos devemos aprender com os problemas, descobrindo como identifica-los e
  corrigi-los. Uma vez descoberto, aplicamos testes automatizados para que eles não
  voltem a acontecer.
  \section{Motivações}
    Como podemos notar, criar um \textit{software} se tornou uma tarefa complexa,
    garantir sua qualidade se tornou uma taréfa difícil, devemos realizar diversas
    ações para assegurar que nosso sistema vai gerar valor. Foi neste cenário que nos
    sentimos motivados. \newline
    Nossas motivações se definem em criar e manter produtos escaláveis,
    consequntemente, possibilitar o controle de seu custo com base na utilização
    dos usuários e na situação da empresa, demonstrando como utilizar a interação
    dos usuários para aprender como melhorar e engajando a utilização de testes
    automatizados como um processo de aprendizagem, detecção de falhas e de
    controle de qualidade.


  \epigrafe{%
    \emph{"Truth can only be found in one place: the code."}
    \begin{flushright}
      -{\cite{CleanCode}}
    \end{flushright}
  }

  \begin{resumo}
  \end{resumo}

  \begin{abstract}
  \end{abstract}

  \listadefiguras

  \listadetabelas

  \sumario

  \chapter{Introdução}
  \capepigrafe[0.5\textwidth]{``A confiança perdida é difícil de recuperar.
  Ela não cresce como as unhas.''}{Brahms, Johannes}

  Com o passar do tempo, os \textit{softwares} firacam cada vez mais importantes
  na sociedade e os negócios começaram a depender ainda mais deles. Alem dos
  computadores, há diversos outros dispositovos que nos permitem acesso a internet
  como por exemplo os celulares, os \textit{tablets}, \textit{video games},
  as televisões, e entre vários outros dispositivos. O que facilitou o acesso a
  informação e consequntemente, a divulgação da informação. Com os sistemas ficando
  mais importantes, diversos serviços são realizados pela internet, como compras,
  negociações, comunicação, tranferências bancarias, entre outros.\newline
  Devido a isso, questões como quantidade de acesso, tempo de resposta e segurnaça
  da informação começaram a ficar cada vez mais importantes na concepção de um
  \textit{software}. A queda de um sistema por alguns segundos, pode ocasionar
  em diversos problemas para uma empresa como, a perda de milhões de reais,
  a desvalorização da marca e causar uma reputação negativa para a empresa.
  Dependendo do motivo da queda, pode ocasionar o fim do sistema e por consequência
  o fim de um produto para a empresa. Outro ponto a ressaltar é que devemos entender
  se mesmo com o sistema funcionando corretamente, ele realmente está trazendo retornos
  positivos para a empresa? O usuário está gostando do que disponibilizamos para ele?
  Está falando bem ou mal do produto? \newline
  Para se preparar para estas situações inesperadas e manter o \textit{software}
  funcionando, é importante manter o produto escalável. Não podemos controlar a
  quantidade de acessos no sistema, mas podemos controlar o quanto de recursos
  computacionais é disponibilizado para o \textit{software}, em épocas em que os
  acessos aumentam, podemos aumentar a quantidade de processamento, em épocas que
  diminuem, podemos diminuir o processamento, assim controlamos o custo do sistema
  e garantimos sua disponibilidade. Não podemos tambem ter um \textit{feedback}
  de todos os usuário de como está a experiência na utilização, muitas vezes nem
  da maioria deles, mas podemos analisar as ações dos usuários para entender
  como está sendo a sua experiência. \newline
  Outro ponto que não deve ser esquecido, é de que problemas vão acontecer,
  situações inesperadas que não vamos saber lidar no momento, brechas na segurança,
  um \textit{bug} no sistema, usuários perdidos na navegação, entre outros casos.
  Nestes casos devemos aprender com os problemas, descobrindo como identifica-los e
  corrigi-los. Uma vez descoberto, aplicamos testes automatizados para que eles não
  voltem a acontecer.
  \section{Motivações}
    Como podemos notar, criar um \textit{software} se tornou uma tarefa complexa,
    garantir sua qualidade se tornou uma taréfa difícil, devemos realizar diversas
    ações para assegurar que nosso sistema vai gerar valor. Foi neste cenário que nos
    sentimos motivados. \newline
    Nossas motivações se definem em criar e manter produtos escaláveis,
    consequntemente, possibilitar o controle de seu custo com base na utilização
    dos usuários e na situação da empresa, demonstrando como utilizar a interação
    dos usuários para aprender como melhorar e engajando a utilização de testes
    automatizados como um processo de aprendizagem, detecção de falhas e de
    controle de qualidade.


  \part{Fundamentos conceituais}

    \chapter{Análise geral sobre desenvolvimento de \textit{software}}

      \section{Metodologias ágeis}

      \section{DevOps}

      \section{CI / CD}

    \chapter{Qualidade de software}

      \section{Requisitos funcionais}

      \section{Requisitos não funcionais}

      \section{Requisitos inversos}

      \section{Escalabilidade}

    \chapter{Testes no desenvolvimento de \textit{software}}

      \section{Testes unitarios}

      \section{Testes de regrssão}

      \section{Testes de fumaça}

      \section{Testes de integração}

      \section{Testes automatizados}

  \chapter{Introdução}
  \capepigrafe[0.5\textwidth]{``A confiança perdida é difícil de recuperar.
  Ela não cresce como as unhas.''}{Brahms, Johannes}

  Com o passar do tempo, os \textit{softwares} firacam cada vez mais importantes
  na sociedade e os negócios começaram a depender ainda mais deles. Alem dos
  computadores, há diversos outros dispositovos que nos permitem acesso a internet
  como por exemplo os celulares, os \textit{tablets}, \textit{video games},
  as televisões, e entre vários outros dispositivos. O que facilitou o acesso a
  informação e consequntemente, a divulgação da informação. Com os sistemas ficando
  mais importantes, diversos serviços são realizados pela internet, como compras,
  negociações, comunicação, tranferências bancarias, entre outros.\newline
  Devido a isso, questões como quantidade de acesso, tempo de resposta e segurnaça
  da informação começaram a ficar cada vez mais importantes na concepção de um
  \textit{software}. A queda de um sistema por alguns segundos, pode ocasionar
  em diversos problemas para uma empresa como, a perda de milhões de reais,
  a desvalorização da marca e causar uma reputação negativa para a empresa.
  Dependendo do motivo da queda, pode ocasionar o fim do sistema e por consequência
  o fim de um produto para a empresa. Outro ponto a ressaltar é que devemos entender
  se mesmo com o sistema funcionando corretamente, ele realmente está trazendo retornos
  positivos para a empresa? O usuário está gostando do que disponibilizamos para ele?
  Está falando bem ou mal do produto? \newline
  Para se preparar para estas situações inesperadas e manter o \textit{software}
  funcionando, é importante manter o produto escalável. Não podemos controlar a
  quantidade de acessos no sistema, mas podemos controlar o quanto de recursos
  computacionais é disponibilizado para o \textit{software}, em épocas em que os
  acessos aumentam, podemos aumentar a quantidade de processamento, em épocas que
  diminuem, podemos diminuir o processamento, assim controlamos o custo do sistema
  e garantimos sua disponibilidade. Não podemos tambem ter um \textit{feedback}
  de todos os usuário de como está a experiência na utilização, muitas vezes nem
  da maioria deles, mas podemos analisar as ações dos usuários para entender
  como está sendo a sua experiência. \newline
  Outro ponto que não deve ser esquecido, é de que problemas vão acontecer,
  situações inesperadas que não vamos saber lidar no momento, brechas na segurança,
  um \textit{bug} no sistema, usuários perdidos na navegação, entre outros casos.
  Nestes casos devemos aprender com os problemas, descobrindo como identifica-los e
  corrigi-los. Uma vez descoberto, aplicamos testes automatizados para que eles não
  voltem a acontecer.
  \section{Motivações}
    Como podemos notar, criar um \textit{software} se tornou uma tarefa complexa,
    garantir sua qualidade se tornou uma taréfa difícil, devemos realizar diversas
    ações para assegurar que nosso sistema vai gerar valor. Foi neste cenário que nos
    sentimos motivados. \newline
    Nossas motivações se definem em criar e manter produtos escaláveis,
    consequntemente, possibilitar o controle de seu custo com base na utilização
    dos usuários e na situação da empresa, demonstrando como utilizar a interação
    dos usuários para aprender como melhorar e engajando a utilização de testes
    automatizados como um processo de aprendizagem, detecção de falhas e de
    controle de qualidade.


  \chapter{Introdução}
  \capepigrafe[0.5\textwidth]{``A confiança perdida é difícil de recuperar.
  Ela não cresce como as unhas.''}{Brahms, Johannes}

  Com o passar do tempo, os \textit{softwares} firacam cada vez mais importantes
  na sociedade e os negócios começaram a depender ainda mais deles. Alem dos
  computadores, há diversos outros dispositovos que nos permitem acesso a internet
  como por exemplo os celulares, os \textit{tablets}, \textit{video games},
  as televisões, e entre vários outros dispositivos. O que facilitou o acesso a
  informação e consequntemente, a divulgação da informação. Com os sistemas ficando
  mais importantes, diversos serviços são realizados pela internet, como compras,
  negociações, comunicação, tranferências bancarias, entre outros.\newline
  Devido a isso, questões como quantidade de acesso, tempo de resposta e segurnaça
  da informação começaram a ficar cada vez mais importantes na concepção de um
  \textit{software}. A queda de um sistema por alguns segundos, pode ocasionar
  em diversos problemas para uma empresa como, a perda de milhões de reais,
  a desvalorização da marca e causar uma reputação negativa para a empresa.
  Dependendo do motivo da queda, pode ocasionar o fim do sistema e por consequência
  o fim de um produto para a empresa. Outro ponto a ressaltar é que devemos entender
  se mesmo com o sistema funcionando corretamente, ele realmente está trazendo retornos
  positivos para a empresa? O usuário está gostando do que disponibilizamos para ele?
  Está falando bem ou mal do produto? \newline
  Para se preparar para estas situações inesperadas e manter o \textit{software}
  funcionando, é importante manter o produto escalável. Não podemos controlar a
  quantidade de acessos no sistema, mas podemos controlar o quanto de recursos
  computacionais é disponibilizado para o \textit{software}, em épocas em que os
  acessos aumentam, podemos aumentar a quantidade de processamento, em épocas que
  diminuem, podemos diminuir o processamento, assim controlamos o custo do sistema
  e garantimos sua disponibilidade. Não podemos tambem ter um \textit{feedback}
  de todos os usuário de como está a experiência na utilização, muitas vezes nem
  da maioria deles, mas podemos analisar as ações dos usuários para entender
  como está sendo a sua experiência. \newline
  Outro ponto que não deve ser esquecido, é de que problemas vão acontecer,
  situações inesperadas que não vamos saber lidar no momento, brechas na segurança,
  um \textit{bug} no sistema, usuários perdidos na navegação, entre outros casos.
  Nestes casos devemos aprender com os problemas, descobrindo como identifica-los e
  corrigi-los. Uma vez descoberto, aplicamos testes automatizados para que eles não
  voltem a acontecer.
  \section{Motivações}
    Como podemos notar, criar um \textit{software} se tornou uma tarefa complexa,
    garantir sua qualidade se tornou uma taréfa difícil, devemos realizar diversas
    ações para assegurar que nosso sistema vai gerar valor. Foi neste cenário que nos
    sentimos motivados. \newline
    Nossas motivações se definem em criar e manter produtos escaláveis,
    consequntemente, possibilitar o controle de seu custo com base na utilização
    dos usuários e na situação da empresa, demonstrando como utilizar a interação
    dos usuários para aprender como melhorar e engajando a utilização de testes
    automatizados como um processo de aprendizagem, detecção de falhas e de
    controle de qualidade.


  \bibliographystyle{../styles/bibliographyStyles/abntex2-alf}

  \bibliography{bibliography}

\end{document}
