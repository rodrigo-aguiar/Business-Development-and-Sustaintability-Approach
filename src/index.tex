\documentclass[]{../class/politex}

\usepackage[num, alf]{abntex2cite}
\usepackage[utf8]{inputenc}
\usepackage{amsmath,amsthm,amsfonts,amssymb}
\usepackage{graphicx,cite,enumerate}
\usepackage[brazil]{babel}

\graphicspath{{images/}}

\titulo{Abordagem para o desenvolvimento de softwares escaláveis focando em
disponibilidade e detecção de falhas}

\autor{Rodrigo Aguiar Ordonis da Silva}
\orientador{Reginaldo Arakaki}

\mba{Transformações Digítais}
\departamento{Laboratório de Arquitetura e Redes de Computadores}

\local{São Paulo}
\data{2020}

\begin{document}
  \capa
  \falsafolhaderosto
  \folhaderosto

  \begin{resumo}
  \end{resumo}

  \begin{abstract}
  \end{abstract}

  \listadefiguras

  \listadetabelas

  \sumario

  \chapter{Conclusão}
  \section{Resultados em relação ao objetivo}
  Cada vez mais, os negócios estão passando por transformações, a reestruturação
  dos modelos de negócio estão cada vez mais comuns, para manter um negócio é
  necessário sempre rever a forma de trabalho da empresa e entender se ela está
  aderente com as movimentações do mercado. \newline
  Com esta abordagem, nosso objetivo foi o de auxiliar o desenvolvimento de novos
  produtos, focando na qualidade do que está sendo desenvolvido e apresentando
  formas para garantir a sua disponibilidade e a detecção de falhas. Em relação
  ao que foi prposto, acreditamos que a utilização desta abordagem facilite as
  equipes nas tomadas de decisões sobre como o produto deve ser desenvolvido,
  como garantir que os requisitos estejam sendo assegurados e que os riscos em
  cada implementação estejam evidentes para todos os envolvidos no projeto.\newline
  Nós também acreditamos, que o sistema deve trazer novas visões do negócio, é
  através dos dados que são gerados com a utilizaçao dos usuários que conseguimos
  captar novas visões de como o negócio funciona e desenvolver novas estratégias
  para adquirir maior vantagem competitiva. Com essas novas estratégias, devemos
  mapear novos requisitos e com os novos requisitos aprimorar nosso produto. \newline
  Com a vinda de novos requisitos e a mudança de requisitos já mapeados, conseguimos
  avaliar se o nosso produto continua aderente ou se ele já cumpriu o seu objetivo
  e devemos começar o desenvolvimento de um novo. Lembrando, como nosso produto
  está escalável, e nossas funcionalidades separadas em sistemas separados, nosso
  produto agora não é somente um sistema mas sim sistemas, e por mais que vamos
  ter que redefinir a sua estrutura e repensar no seu objetivo, as funcionalidades
  que ainda estão aderentes, deverão continuar no nosso próximo produto, desta
  forma conseguimos maior agilidade no desenvolvimento e conseguimos focar
  diretamente na nova perspectiva da empresa, mantendo aquilo que continua
  consolidado.

  \section{Trabalhos futuros}
  Como futuro trabalho, podemos evoluir esta abordagem como uma métrica focando
  nas questões de arquitetura e em requisitos de qualidade, onde vamos específicar
  mais como garantir a escalabilidade, e apresentando mais fatores de qualidade
  além de disponibilidae de detecção de falhas. Podemos evoluir os processos de
  desenvolvimento apresentados, com mais detalhes e apresentando a abordagem sendo
  utilizada durante projetos reais.


  \chapter{Conclusão}
  \section{Resultados em relação ao objetivo}
  Cada vez mais, os negócios estão passando por transformações, a reestruturação
  dos modelos de negócio estão cada vez mais comuns, para manter um negócio é
  necessário sempre rever a forma de trabalho da empresa e entender se ela está
  aderente com as movimentações do mercado. \newline
  Com esta abordagem, nosso objetivo foi o de auxiliar o desenvolvimento de novos
  produtos, focando na qualidade do que está sendo desenvolvido e apresentando
  formas para garantir a sua disponibilidade e a detecção de falhas. Em relação
  ao que foi prposto, acreditamos que a utilização desta abordagem facilite as
  equipes nas tomadas de decisões sobre como o produto deve ser desenvolvido,
  como garantir que os requisitos estejam sendo assegurados e que os riscos em
  cada implementação estejam evidentes para todos os envolvidos no projeto.\newline
  Nós também acreditamos, que o sistema deve trazer novas visões do negócio, é
  através dos dados que são gerados com a utilizaçao dos usuários que conseguimos
  captar novas visões de como o negócio funciona e desenvolver novas estratégias
  para adquirir maior vantagem competitiva. Com essas novas estratégias, devemos
  mapear novos requisitos e com os novos requisitos aprimorar nosso produto. \newline
  Com a vinda de novos requisitos e a mudança de requisitos já mapeados, conseguimos
  avaliar se o nosso produto continua aderente ou se ele já cumpriu o seu objetivo
  e devemos começar o desenvolvimento de um novo. Lembrando, como nosso produto
  está escalável, e nossas funcionalidades separadas em sistemas separados, nosso
  produto agora não é somente um sistema mas sim sistemas, e por mais que vamos
  ter que redefinir a sua estrutura e repensar no seu objetivo, as funcionalidades
  que ainda estão aderentes, deverão continuar no nosso próximo produto, desta
  forma conseguimos maior agilidade no desenvolvimento e conseguimos focar
  diretamente na nova perspectiva da empresa, mantendo aquilo que continua
  consolidado.

  \section{Trabalhos futuros}
  Como futuro trabalho, podemos evoluir esta abordagem como uma métrica focando
  nas questões de arquitetura e em requisitos de qualidade, onde vamos específicar
  mais como garantir a escalabilidade, e apresentando mais fatores de qualidade
  além de disponibilidae de detecção de falhas. Podemos evoluir os processos de
  desenvolvimento apresentados, com mais detalhes e apresentando a abordagem sendo
  utilizada durante projetos reais.


  \chapter{Conclusão}
  \section{Resultados em relação ao objetivo}
  Cada vez mais, os negócios estão passando por transformações, a reestruturação
  dos modelos de negócio estão cada vez mais comuns, para manter um negócio é
  necessário sempre rever a forma de trabalho da empresa e entender se ela está
  aderente com as movimentações do mercado. \newline
  Com esta abordagem, nosso objetivo foi o de auxiliar o desenvolvimento de novos
  produtos, focando na qualidade do que está sendo desenvolvido e apresentando
  formas para garantir a sua disponibilidade e a detecção de falhas. Em relação
  ao que foi prposto, acreditamos que a utilização desta abordagem facilite as
  equipes nas tomadas de decisões sobre como o produto deve ser desenvolvido,
  como garantir que os requisitos estejam sendo assegurados e que os riscos em
  cada implementação estejam evidentes para todos os envolvidos no projeto.\newline
  Nós também acreditamos, que o sistema deve trazer novas visões do negócio, é
  através dos dados que são gerados com a utilizaçao dos usuários que conseguimos
  captar novas visões de como o negócio funciona e desenvolver novas estratégias
  para adquirir maior vantagem competitiva. Com essas novas estratégias, devemos
  mapear novos requisitos e com os novos requisitos aprimorar nosso produto. \newline
  Com a vinda de novos requisitos e a mudança de requisitos já mapeados, conseguimos
  avaliar se o nosso produto continua aderente ou se ele já cumpriu o seu objetivo
  e devemos começar o desenvolvimento de um novo. Lembrando, como nosso produto
  está escalável, e nossas funcionalidades separadas em sistemas separados, nosso
  produto agora não é somente um sistema mas sim sistemas, e por mais que vamos
  ter que redefinir a sua estrutura e repensar no seu objetivo, as funcionalidades
  que ainda estão aderentes, deverão continuar no nosso próximo produto, desta
  forma conseguimos maior agilidade no desenvolvimento e conseguimos focar
  diretamente na nova perspectiva da empresa, mantendo aquilo que continua
  consolidado.

  \section{Trabalhos futuros}
  Como futuro trabalho, podemos evoluir esta abordagem como uma métrica focando
  nas questões de arquitetura e em requisitos de qualidade, onde vamos específicar
  mais como garantir a escalabilidade, e apresentando mais fatores de qualidade
  além de disponibilidae de detecção de falhas. Podemos evoluir os processos de
  desenvolvimento apresentados, com mais detalhes e apresentando a abordagem sendo
  utilizada durante projetos reais.


  \chapter{Fundamentos conceituais}

    \section{Análise geral sobre desenvolvimento de \textit{software}}

      \subsection{Metodologias ágeis}

      \subsection{DevOps}

      \subsection{CI / CD}

    \section{Qualidade de software}

      \subsection{Requisitos funcionais}

      \subsection{Requisitos não funcionais}

      \subsection{Requisitos inversos}

      \subsection{Escalabilidade}

    \section{Testes no desenvolvimento de \textit{software}}

      \subsection{Testes unitarios}

      \subsection{Testes de regrssão}

      \subsection{Testes de fumaça}

      \subsection{Testes de integração}

      \subsection{Testes automatizados}

  \chapter{Conclusão}
  \section{Resultados em relação ao objetivo}
  Cada vez mais, os negócios estão passando por transformações, a reestruturação
  dos modelos de negócio estão cada vez mais comuns, para manter um negócio é
  necessário sempre rever a forma de trabalho da empresa e entender se ela está
  aderente com as movimentações do mercado. \newline
  Com esta abordagem, nosso objetivo foi o de auxiliar o desenvolvimento de novos
  produtos, focando na qualidade do que está sendo desenvolvido e apresentando
  formas para garantir a sua disponibilidade e a detecção de falhas. Em relação
  ao que foi prposto, acreditamos que a utilização desta abordagem facilite as
  equipes nas tomadas de decisões sobre como o produto deve ser desenvolvido,
  como garantir que os requisitos estejam sendo assegurados e que os riscos em
  cada implementação estejam evidentes para todos os envolvidos no projeto.\newline
  Nós também acreditamos, que o sistema deve trazer novas visões do negócio, é
  através dos dados que são gerados com a utilizaçao dos usuários que conseguimos
  captar novas visões de como o negócio funciona e desenvolver novas estratégias
  para adquirir maior vantagem competitiva. Com essas novas estratégias, devemos
  mapear novos requisitos e com os novos requisitos aprimorar nosso produto. \newline
  Com a vinda de novos requisitos e a mudança de requisitos já mapeados, conseguimos
  avaliar se o nosso produto continua aderente ou se ele já cumpriu o seu objetivo
  e devemos começar o desenvolvimento de um novo. Lembrando, como nosso produto
  está escalável, e nossas funcionalidades separadas em sistemas separados, nosso
  produto agora não é somente um sistema mas sim sistemas, e por mais que vamos
  ter que redefinir a sua estrutura e repensar no seu objetivo, as funcionalidades
  que ainda estão aderentes, deverão continuar no nosso próximo produto, desta
  forma conseguimos maior agilidade no desenvolvimento e conseguimos focar
  diretamente na nova perspectiva da empresa, mantendo aquilo que continua
  consolidado.

  \section{Trabalhos futuros}
  Como futuro trabalho, podemos evoluir esta abordagem como uma métrica focando
  nas questões de arquitetura e em requisitos de qualidade, onde vamos específicar
  mais como garantir a escalabilidade, e apresentando mais fatores de qualidade
  além de disponibilidae de detecção de falhas. Podemos evoluir os processos de
  desenvolvimento apresentados, com mais detalhes e apresentando a abordagem sendo
  utilizada durante projetos reais.


  \chapter{Resultados da proposta}

    \section{Um produto escalável}

      \subsection{Um produto com custo dinâmico}

    \section{Um produto disponível}

    \section{Um produto com falhas planejadas}

  \chapter{Conclusão}

    \section{Resultados em relação ao objetivo}

    \section{Trabalhos futuros}

  \bibliographystyle{abntex2-num}

  \bibliography{bibliography}

\end{document}
